\chapter{列方程解应用题二}
\section{知识点}
列方程的实质是把题中的“生活语言”化为“代数语言”,即把文字等量关系式
用已知数与未知数代入即得方程。\par
列方程解应用题的两个关键点:
\begin{enumerate}
    \item 用x 表示未知量
    \item 建立等量关系
\end{enumerate}
\vspace{-0.8cm}
\section{例题分析}
\begin{example}
    某车间生产甲、乙两种零件,生产的甲种零件比乙种零件多12 个,乙种零件
全部合格,甲种零件只有$\dfrac{4}{5}$合格,两种零件合格的一共是42 个,两种零件各生产
了多少个?
\end{example}
\vspace{3cm}
\begin{example}
    袋子里有红、黄、蓝三种颜色的球,黄球个数是红球的
$\dfrac{4}{5}$,蓝球个数是红球
的
$\dfrac{2}{3}$ ,黄球个数的
$\dfrac{3}{4}$ 比蓝球少2 个。袋中共有多少个球?
\end{example}
\vspace{3cm}
\begin{example}
    有一个水池,第一次放出全部水的
$\dfrac{2}{5}$,第二次放出30 立方米水,第三次又放
出剩下水的$\dfrac{2}{5}$,池里还剩水54 立方米,全池蓄水为多少立方米?
\end{example}
\vspace{3cm}
\section{练习}
\vspace{-0.5cm}
\begin{exercise}
    甲、乙两人共有存款108 元,如果甲取出自己存款的
$\dfrac{2}{5}$ ,乙取出12 元后,两人
所存的钱数相等,甲、乙两人原来各有存款多少元?
\end{exercise}
\vspace{3cm}
\begin{exercise}
    六年级有学生300 人,从六年级男生中选出
$\dfrac{3}{4}$ ,女生中选出
$\dfrac{1}{2}$ 参加校运动会,
这样全年级还剩下91 人参加布置会场工作。六年级有男、女生各多少人?,
\end{exercise}
\vspace{3cm}
\begin{exercise}
    长江文具店运来的毛笔比钢笔多1000 支,其中毛笔的
$\dfrac{3}{7}$和钢笔的
$\dfrac{1}{2}$相等,长江
文具店共运来多少支笔?
\end{exercise}
\vspace{3cm}
\begin{exercise}
    某人装修房屋,原预算25000 元。装修时因材料费下降了20$\%$,工资涨了10$\%$,
实际用去21500 元。求原来材料费及工资各是多少元?
\end{exercise}
\vspace{3cm}
\begin{exercise}
    某商店因换季销售某种商品,如果按定价的5 折出售,将赔30 元,按定价的9
折出售,将赚20 元,则商品的定价为多少元?
\end{exercise}
\vspace{3cm}
\begin{exercise}
    某书店出售一种挂历,每售出1 本可得18 元利润。售出一部分后每本减价10
元出售,全部售完。已知减价出售的挂历本数是减价前出售挂历本数的
$\dfrac{2}{3}$。书店售
完这种挂历共获利润2870 元。书店共售出这种挂历多少本?
\end{exercise}
\vspace{3cm}
\begin{exercise}
    甲、乙两人各有钱若干,现有18 元奖金,如果全部给甲,则甲的钱为乙的2
倍,如果全部给乙,则乙的钱为甲的
$\dfrac{7}{8}$ 。问原来两人各有多少元钱?
\end{exercise}
\vspace{3cm}
\begin{exercise}
    一位牧羊人赶着一群羊去放牧,跑出一只公羊后,他数了数羊的只数,发现
剩下的羊中,公羊与母羊的只数比是9:7;过了一会跑走的公羊又回到了羊群,却
又跑走了一只母羊,牧羊人又数了数羊的只数,发现公羊与母羊的只数比是7:5。
这群羊原来有多少只?
\end{exercise}
