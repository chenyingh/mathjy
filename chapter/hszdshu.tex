\chapter{指数函数与对数函数}
\section{指数运算与对数运算}
\begin{enumerate}
	\item 指数的性质
	\begin{enumerate}
		\item $a^n=a·a·a......(a\in R)$
		\item $a^{-n}=\frac{1}{a^n} (a\neq 0)$
		\item  $a^0=1 (a\neq 0)$
		 \item $a^{\frac{m}{n}}=\sqrt[n]{m}$
	\end{enumerate}
\item  指数的运算
  \begin{enumerate}
  	\item $(a^m)^n=a^{(m·n)}$
  	\item $a^m·a^n=a^{(m+n)}$
  	\item $(a·b)^n=a^n·a^b$
  \end{enumerate}
\item 对数的性质
     \begin{enumerate}
     	\item $a^b \Longleftrightarrow b=\log_aN (a>0,a\neq 0)$
     	\item $a^0=1\Longrightarrow \log_a1=0$
     	\item  $a^1=a \Longrightarrow \log_a a$ 
     	\item $a^{\log_a x}=x$
     	\item $\log_a a^x=x$
     	\item $log_ab=\dfrac{\log_c b}{\log_c a} \Longrightarrow (\log_a b=\dfrac{1}{\log_b a};\log_a b=\log_{a^N}b^n)$
     	\item  $\log_{a^N}b=\frac{1}{n}\log_ab$
     \end{enumerate}
 \item 对数的运算
 \begin{enumerate}
 	\item $\log_a(M·N)=\log_a M +\log_a N$
 	\item $\log_a(\frac{M}{N})=\log_a M-\log_aN$
 	\item $\log_aM^n=nlog_a M$
 \end{enumerate}
 
\end{enumerate}
\section{例题分析}
\begin{example} 比较下来各数的大小.\\
	(1)$0.3^2\hspace{0.5cm} \log_2\frac{1}{3}\hspace{0.5cm}\log_3\frac{1}{9} \hspace{0.5cm} \log_{\frac{1}{2}}5\hspace{0.5cm} \log_4 15 \hspace{0.5cm} \log_{\frac{1}{2}}{\frac{1}{15}}	$~\vspace*{0.6cm}\\
	(2)$\log_{0.5}0.6 \hspace{0.5cm} \log_{\sqrt{2}}0.5 \hspace{0.5cm} \log_{\sqrt{3}}\sqrt{5}$\\
\end{example}

\begin{example}求下列函数的单调性、最值\par
	$(1)y=3^{|x|}-1$  \hfil $y=2^{x^2-x+1}$\par
	\vspace*{3cm}
	$(3)y=4^x-x^{x+1}+1$\hfil $(4)y=\log_a(x^2-1)$\par
	\vspace*{3.5cm}
	$(5)y=\log_2(x^2-3x+2)$\hfil $(6)y=(\log_2x)^2-3\log_2x+2,x\in[1,2]$
	\vspace*{3.5cm}
\end{example}
\begin{example}判断下列函数的奇偶性\par
$(1)f(x)=\dfrac{1}{2^x-1}+\dfrac{1}{2}$\par
\vspace{3cm}
$(2)f(x)=\lg\dfrac{1-x}{1+x}$\par
\vspace{3cm}
$(3)f(x)=\ln(\sqrt{1+x^2}-x)$\par
\vspace{3cm}
\end{example}
\begin{example}已知函数$f(x)=\lg(ax^2+2x+1)$\par
	(1)若$f(x)$的定义域为$R$,求实数a的取值范围.\par
	\vspace{4cm}
	(2)若$f(x)$的值域为$R$,求实数$a$的取值范围.\par
	\vspace{4cm}
\end{example}
\begin{example}分别求函数$f(x)=2^x+2^{-x}$和$g(x)=\dfrac{1-2^x}{4^x}$的定义域、值域\par

\end{example}
% Table generated by Excel2LaTeX from sheet 'Sheet1'
\begin{table}
	\centering
	\caption{TableName}
	\begin{tabular}{|l|l|l|l|}
		\hline
		
		Id & Name & Age & Gender\\ \hline
		1 & Lawrence & 39 & M\\ \hline
		2 & Oliver & 25 & M\\ \hline
		3 & Roberta & 17 & F\\ \hline
		4 & Bamboo & 70 & F\\ \hline
		
	\end{tabular}
\end{table}

