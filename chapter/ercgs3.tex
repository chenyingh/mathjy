\chapter{二次根式的加法}
\section{二次根式的加法法则}
\begin{enumerate}
    \item  判断二次根式是否可以合并的方法:\begin{enumerate}
        \item 先将二次根式化成最简二次根式;
        \item 在看被开方数是否相同.
    \end{enumerate}
    \item 法则$a\sqrt{m}+b\sqrt{m}=(a+b)\sqrt{m}$.(1.根号外的因数相加;2.根指数和被开方数不变) 
\end{enumerate}
\section{例题分析}
\begin{example}计算.\par
    (1).$\sqrt{80}-\sqrt{45}$ \hfil (2)$\sqrt{9 a}+\sqrt{25 a}$.\par
    \vspace{2cm}
    (3).$\sqrt{8}+\sqrt{\dfrac{1}{50}}$ \hfil (4).$3 \sqrt{12}-\sqrt{\dfrac{1}{27}}$\par
    \vspace{2cm}
    (5).$\sqrt{3}+\sqrt{12}+\sqrt{27}+\sqrt{5}$ \hfil (6).$\dfrac{1}{2} \sqrt{12}-\left(3 \sqrt{\dfrac{1}{3}}+\sqrt{2}\right)$\par
    \vspace{2cm}
    (7).$3 \sqrt{8}-\dfrac{1}{2} \sqrt{32}+7 \sqrt{\dfrac{1}{8}}$\hfil (8).$\dfrac{2}{3} \sqrt{9 x}+6 \sqrt{\dfrac{x}{4}}-2 x \sqrt{\dfrac{1}{x}}$
\vspace{2cm}
\end{example}
\begin{example}混合运算.\par
    (1).$\left(\sqrt{\dfrac{1}{3}}+\sqrt{27}\right) \times \sqrt{3}$ \hfil (2).$(4 \sqrt{3}-3 \sqrt{6}) \div 2 \sqrt{3}$ \par
    \vspace{2cm}
    (3).$(5+\sqrt{7})(5-\sqrt{7})$ \hfil (4).$(\sqrt{5}+1)(\sqrt{5}-1)-\sqrt{18} \times \sqrt{3}+\dfrac{\sqrt{2}}{\sqrt{3}}$
\vspace{2cm}   
\end{example}
\begin{example}
    已知最简二次根式$\sqrt{2 a+1}$与$\sqrt{3-2 a}$是可以合并的,求a的值.
\end{example}
\vspace{2cm}
\begin{example}计算.\par
    (1).$(\sqrt{3}+\sqrt{2}-\sqrt{6})^{2}-(\sqrt{2}-\sqrt{3}+\sqrt{6})^{2}$\par
    \vspace{2cm}
    (2).$(1+\sqrt{2}-\sqrt{3})(1-\sqrt{2}+\sqrt{3})$   \par
  \vspace{2cm}
    (3). $(2 \sqrt{2}-\sqrt{7})^{2017} \cdot(2 \sqrt{2}+\sqrt{7})^{2018}$
\end{example}
\vspace{2cm}
\begin{example}
    已知$x=7+4 \sqrt{3}$,$y=7-4 \sqrt{3}$,求$\dfrac{x}{y}+\dfrac{y}{x}$的值.
\end{example}
\vspace{2cm}
\begin{example}
    一个三角形的边长分别为$5\sqrt{\dfrac{x}{5}}$, $\dfrac{1}{2} \sqrt{20 x}$, $\dfrac{5}{4} x \sqrt{\dfrac{4}{5 x}}$\par
    (1).求它的周长.\par
     (2).请你给$x$一个合适的值,使三角形的周长为整数,并求出此时三角形的周长的值.
\end{example}
\vspace{2cm}
\begin{example}已知$x=2-\sqrt{3}$,$y=x+\sqrt{3}$求下列代数式的值:\par
        (1).$\sqrt{x^2+y^2}$.\par
    (2).$x^2-y^2$
\end{example}

