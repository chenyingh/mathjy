\chapter{图示法解分数应用题}
\section{知识点}
图示法就是用线段图(或其它图形)把题目中的已知条件和问题表示出来,这
样可以把抽象的数量关系具体化,往往可以从图中找到解题的突破口。运用图示法
教学应用题,是培养思维能力的有效方法之一。\par
图示法不仅可以形象地、直观地反映分数应用题中的“对应量和对应分率”间
的关系,启发学生的解题思路,帮助学生找到解题的途径,而且通过画图的训练,
可以调动学生思维的积极性,提高学生分析问题和解决问题的能力。
\section{典型例题}
\begin{example}
    一条鱼重的$\frac{3}{5}$加上$\frac{3}{4}$千克就是这条鱼的重量,这条鱼重多少千克?
\end{example}
\vspace{2.5cm}
\begin{example}
    一桶油第一次用去$\frac{1}{5}$,第二次比第一次多用去20 千克,还剩下22 千克。原
来这桶油有多少千克?
\end{example}
\vspace{2.5cm}
\begin{example}
    缝纫机厂女职工占全厂职工人数的$\frac{7}{20}$,比男职工少144 人,缝纫机厂共有职
工多少人?
\end{example}
\vspace{2.5cm}
\begin{example}
    张亮从甲城到乙城,第一天行了全程的40\%,第二天行了全程的$\frac{9}{20}$,距乙城还
    有18 千米,甲、乙两城相距多少千米?
\end{example}
\vspace{2.5cm}
\begin{example}
    李玲看一本书,第一天看了全书的$\frac{1}{6}$,第二天看了18 页,这时正好看了全书的
一半。李玲第一天看书多少页?
\end{example}
\vspace{2.5cm}
\begin{example}
    某工程队修筑一条公路,第一周修了这段公路的$\frac{1}{4}$,第二周修了这段公路的
$\frac{2}{7}$。第二周比第一周多修了2 千米,这段公路全长多少千米?
\end{example}
\vspace{2.5cm}
\begin{example}
    汽车从学校出发到太湖玩,$\frac{6}{7}$小时行驶了全程的$\frac{3}{4}$
,这时距太湖边还有4 千米。
照这样的速度,行完全程共用多少小时?
\end{example}
\vspace{2.5cm}
\begin{example}
    某书店运来一批连环画。第一天卖出1800 本,第二天卖出的本数比第一天多
$\frac{1}{9}$,余下总数的$\frac{3}{7}$ 正好第三天全部卖完,这批连环画共有多少本?
\end{example}
\vspace{2.5cm}
\begin{example}
    一辆汽车从甲地开往乙地,第1 小时行了$\frac{1}{7}$,第2 小时比第1 小时少行了16 千
米,这时汽车距甲地94 千米。甲、乙两地相距多少千米?
\end{example}
\vspace{2.5cm}
\begin{example}
    水果店购进一批水果,第一天卖了30\%,第二天卖出余下的50\%,这两天共
卖出195 千克。这批水果共多少千克?
\end{example}
\vspace{2.5cm}
\begin{example}
    用绳子测井深,把绳子折成三股来量,井外余$\frac{4}{3}$米,把绳子折成四股来量,
井外余$\frac{1}{3}$米,井深多少米?
\end{example}
\vspace{2.5cm}