\chapter{浓度问题}
\section{知识点}
在百分数应用题中有一类叫溶液配比问题, 即浓度问题。如将糖溶于水就得到
了糖水,其中糖叫溶质,水叫溶剂,糖水叫溶液。如果水的量不变,那么糖加的越
多,糖水就越甜,也就是说糖水甜的程度是由糖(溶质)与糖水(溶液=糖+水)二
者质量的比值决定的。这个比值就叫糖水的含糖量或糖含量。即:
$$ \text{浓度}=\dfrac{\text{溶质质量}}{\text{溶液质量}}\times 100\%=\dfrac{\text{溶质质量}}{\text{溶质质量+溶剂质量}}$$\par
浓度问题变化多,有些题目的难度较大,计算也比较复杂。要根据题目的条件
和问题逐一分析,也可以分步解答,也可以列方程解答。
\section{例题分析}
\begin{example}
    现有浓度为20$\%$的糖水20 千克,要得到浓度为10$\%$的糖水,需加水多少千
    克?
\end{example}
\vspace{2cm}
\begin{example}
    一容器内有浓度为25$\%$的糖水,若再加入20 千克水,糖水的浓度变为15$\%$。
这个容器内原来含有糖多少千克?
\end{example}
\vspace{2cm}
\begin{example}要配制浓度为20$\%$的盐水1000 克,需浓度为10$\%$和浓度为30$\%$的盐水各多
少克?
\end{example}
\vspace{2cm}
\begin{exercise}
    
\end{exercise}