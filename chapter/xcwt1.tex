\chapter{行程问题之巧用比例}
\section{知识点}
行程问题常和比例结合起来,题目虽然简洁,但是综合性强,而且形式多变,
运用比例知识解决复杂的行程问题经常考,而且要考都不简单。\par
我们知道行程问题里有三个量:速度、时间、距离,知道其中两个量就可以求
出第三个量。速度$\times$时间=距离;距离$\div$速度=时间;距离$\div$时间=速度。如果要用比例
做行程问题,这三个量之间的关系是:\par
\begin{enumerate}
\item 时间相同,速度比=距离比;
\item  速度相同,时间比=距离比;
\item  距离相同,速度比=时间的反比。
\end{enumerate}
\section{典型例题}
\begin{example}
    客车和货车同时从甲、乙两城之间的中点向相反的方向行驶, 3 小时后,客车
到达甲城,货车离乙城还有30 千米.已知货车的速度是客车的3/4,甲、乙两城相
距多少千米?
\end{example}
\vspace{2.5cm}
\begin{example}
    甲、乙两车同时从A、B 两地相向而行,它们相遇时距A 、B 两地中心处8
千米,已知甲车速度是乙车的1.2 倍,求A、B 两地的距离。
\end{example}
\vspace{2.5cm}
\begin{example}
    甲、乙两车分别从A 、B 两地同时出发,相向而行,甲车每小时行48 千米,
乙车每小时行42 千米。当乙车行至全程的$\frac{7}{20}$时,甲车距中点还有24 千米, A 、B
两地相距多少千米?
\end{example}
\vspace{2.5cm}
\begin{example}
    甲、乙两辆汽车同时从A、B 两地相向而行,甲行到全程的$\frac{3}{7}$的地方与乙相遇。
甲每小时行30 千米,乙行完全程需7 小时。求A、B 两地之间的路程。
\end{example}
\vspace{2.5cm}
\begin{example}
    一列货车和一列客车同时从甲乙两地相向开出, 已知客车的速度是货车的速度的
$\frac{2}{3}$,两车相遇时,客车比货车少行8 千米。求甲、乙两地间的距离。
\end{example}
\vspace{2.5cm}
\begin{example}
    甲、乙两车分别从A 、B 两地同时出发,相向而行,甲车每小时行56 千米,乙
车每小时行40 千米。当乙车行至全程的$\frac{2}{5}$时,甲车已超过中点12 千米, A 、B 两
地相距多少千米?
\end{example}
\vspace{2.5cm}
\begin{example}
    甲、乙两人分别从A、B 两地相向而行,甲行了全程的$\frac{5}{11}$ ,正好与乙相遇,已
    知甲每小时行4.5 千米,乙行完全程要5.5 小时,求A、B 两地相距多少千米?
\end{example}
\vspace{2.5cm}
\begin{example}
    客车和货车同时从A、B 两地相对开出, 货车的速度是客车的$\frac{2}{3}$ 。两车在离两地
中点30 千米处相遇。A、B 两地相距多少千米?
\end{example}
\vspace{2.5cm}
\begin{example}
    甲、乙两车分别从A、B 两地同时出发,相向而行,甲车速度是乙车速度的
$\frac{5}{6}$当乙车行至全程的$\frac{2}{5}$时,甲车距中点还有30 千米, A、B 两地相距多少千米?
\end{example}
\vspace{2.5cm}