\chapter{直线与方程}
\section{直线的倾斜角与斜率}
\subsection{倾斜角与斜率}
\begin{definition}{}
	当直线$l$与$x$相交时,直线$l$向上的方向与$x$轴的正方向所成的角$\alpha $叫做直线$l$的\textbf{倾斜角}.倾斜角$\alpha$的正切值叫做这条直线的\textbf{斜率}.斜率用$k$表示.$\alpha=90^\circ $时斜率不存在.
	$$k=\tan \alpha  (o^\circ \leq \alpha <180^\circ)	$$
	经过两点$P_1(x_1,y_1),P_2(x_2,y_2)(x_1\neq x_2)$的直线斜率的公式为:\\
	$$k=\dfrac{y_2-y_1}{x_2-x_1}$$
\end{definition}
\subsection{两条直线平行与垂直的判定}
对于两条直线$l_1,l_2$,其斜率分别为$k_1,k_2$,
$k_1=k_2 \Leftrightarrow l_1 // l_2$或$l_1$与$l_2$重合.$k_1 k_2=-1\Leftrightarrow l_1 \perp l_2$.
\section{直线的方程}
\subsection{直线的点斜式}
$$y-y_0=k(x-x_0)\text{(点斜式)}. y=kx+b\text{(斜截式)}.$$
\begin{note}
	当直线l的倾斜角为$0^\circ$时,$\tan 0^\circ=0$,所以方程为:$y=y_0$.\\
	当直线l的倾斜角为$90^\circ$时,直线没有斜率,方程为:$x=x_0$
\end{note}
\subsection{直线的两点式方程}
\begin{equation}
\dfrac{y-y_1}{y_2-y_1}=\dfrac{x-x_1}{x_2-x_1}\text{两点式}\\
\dfrac{x}{a}+\dfrac{y}{b}=1\text{截距式}
\end{equation}
\subsection{直线的一般式方程}
$$Ax+By+C=0$$
\subsection{两条直线的交点坐标}
\subsection{两点间的距离}
$P_1(x_1,y_1),P_2(x_2,y_2)$间的距离公式\\
$$|P_1P_2|=\sqrt{(x_2-x_1)^2+(y_2-y_1)^2}$$
\subsection{点到直线的距离}
点$P_0(x_0,y_0)$到直线$l:Ax+Bx+C=0$的距离公式为:\\
$$d=\dfrac{|Ax_0+By_0+C|}{\sqrt{A^2+B^2}}$$
\subsection{两条平行线的距离}
$l_1:Ax+By+C_1=0$和$l_2:Ax+By+C_2=0$的距离公式为:\\
$$d=\dfrac{|C_1-C_2|}{\sqrt{A^2+B^2}}$$

