\chapter{函数的奇偶性}
\section{例题分析}
\begin{example}
	已知偶函数$f(x)$在区间[0,4]上是增函数,试比较$f(-3)$与$f(\pi)$的大小.
\end{example}
\vspace{2.5cm}
\begin{example}
	若奇函数$f(x)$在[3,7]上的最小值是5,那么$f(x)$在[-7,-3]上(\qquad)\par
	A.最小值是5\hfil B.最小值是-5  \hfil  C.最大值是-5   \hfil  D.最大值是5
\end{example}
\begin{example}
$f(x)$是定义在$R$上的奇函数,又$f(x)$在区间(0,+$\infty$)上是增函数,且$f(1)=0$,则满足$f(x)>0$的x的取值范围集合是(\qquad)
\end{example}
\vspace{2.5cm}
\begin{example}
设$f(x)$是定义在$R$上的奇函数,又$f(x+2)=-f(x)$,当$0\leq x \leq 1$时,$f(x)=x$,则$f(7.5)$等于(\qquad) 
\end{example}
\vspace{2.5cm}
\begin{example}
如果函数$f(x)$在$R$上为奇函数,在[-1,0)上是增函数,且$f(x+2)=-f(x)$,试比较$f(\frac{1}{3}),f(8.5),f(1)$的大小关系(\qquad)
\end{example}
\vspace{2.5cm}
\begin{example}
若$f(x)$为奇函数,且在(0,+$\infty$)内是增函数,又$f(-3)=0$,则$xf(x)<0$的解集为(\qquad)
\end{example}
\vspace{2.5cm}
\begin{example}
设函数$y=f(x)$对于任意的$x,y\in R$都有$f(x+y)=f(x)+f(y)$,且f(x)不恒为零,判断f(x)的奇偶性.
\end{example}
\vspace{2.5cm}
\begin{example}
f(x)是定义在$R$上的偶函数,且当$x\leq 0$时,$f(x)=x^2-x$,求$f(x)$的解析式.并画出函数图象,求出函数的值域。
\end{example}
\vspace{2.5cm}
\begin{example}
已知$f(x)$是定义在$R$上的奇函数,当$x>0$时$f(x)=x^2-4x+3$,求$f(x)$的解析式.
\end{example}
\vspace{2.5cm}
\begin{example}
已知函数$f(x)$是定义在区间[-2,2]上的偶函数,当$x\in [0,2]$时,f(x)是减函数,如果不等式$f(1-m)<f(m)$成立,求实数m的取值范围
\end{example}
\vspace{2.5cm}
\begin{example}
	已知函数f(x)定义域$R$,为对任意的$x_1,x_2\in R$都有$f(x_1+x_2)=f(x_1)+f(x_2)$且$x>0$时$f(x<0)$,f(1)=-2,试判断在区间[-3,3]上$f(x)$是否有最大值和最小值?如果有试求出最大值和最小值,如果没有请说明理由. 
\end{example}
\vspace{2.5cm}
\begin{example}
已知函数$f(x)$定义域$R$,为对任意的$x_1,x_2\in R$都有$f(x_1+x_2)=f(x_1)f(x_2)$且$x>0$时,$0<f(x)<1$,求$f(0)$的值并求$x<0$时$f(x)$的取值范围.
\end{example}
\vspace{2.5cm}
\begin{example}
已知$f(x)$是定义在$R$上的奇函数,当$x\geq$时,$f(x)=x^2-2x$,则f(x)在R上的表达式是\par
$A.f(x)=x^2-2x$\hfil $B.f(x)=x(|x|-1)$\par
$C.f(x)=|x|(x-2)$\hfil $D.f(x)=x(|x|-2)$
\end{example}
\begin{example}
函数$f(x)$是定义在[-6,6]上的偶函数,且在[-6,0]上是减函数,则(\qquad)\par
$A.f(x)+f(4)>0$\hfil $B.f(-3)-f(2)<0$\par
$C.f(-2)+f(-5)<0$\hfil $D.f(4)-f(-1)>0$
\end{example}
\begin{example}
设$f(x)$是定义在$R$上的任意一个增函数,令$F(x)=f(x)-f(-x)$,则$F(x)$必是(\qquad)\par
A.增函数且是奇函数      \hfil    B.增函数且是偶函数 \par
 C.减函数且是奇函数       \hfil   D.减函数且是偶函数
\end{example}
\begin{example}
	设$f(x),g(x)$都是奇函数,$F(x)=f(x)+g(x)+3$,若$F(5)=9$,$F(-5)$=(\qquad)\par
	$A.3$  \hfil $B.-3$  \hfil $C.-6$  \hfil $D.-9$ 
\end{example}
\begin{example}
设$f(x),g(x)$都是奇函数,且$F(x)=af(x)+bg(x)+2$,若在$(0,+\infty)$上$F(x)$有最大值8,则在$(\infty,0)$上$F(x)$有(\qquad)\par
A.最小值-8  \hfil       B.最大值-8  \hfil     C.最小值-4  \hfil         D.最小值-6
\end{example}
\begin{example}
函数$f(x)=\dfrac{\sqrt{1-x^2}}{|x+3|-3}$是(\qquad)\par
A. 奇函数  \hfil  B. 偶函数 \hfil   C. 既奇又偶函数  \hfil D. 没有确定的奇偶性
\end{example}
\begin{example}
若$f(x)=ax^2+bx+c(a\neq 0)$是偶函数,则$g(x)=ax^3+bx^2+cx$是(\qquad)\par
A. 奇函数 \hfil B. 偶函数 \hfil  C. 既奇又偶函数 \hfil   D. 没有确定的奇偶性
\end{example}
\begin{example}
下列命题:(1)偶函数图象一定与y轴相交(2)奇函数图象一定过原点(3)偶函数图象关于y轴对称(4)既奇又偶函数一定满足$f(x)=0$,其中真命题个数是(\qquad)\par
A. 1    \hfil   B. 2   \hfil    C. 3   \hfil     D. 4
\end{example}
\begin{example}
若奇函数$f(x)$在区间[3,7]上是增函数,且最小值为5,那么在区间[-7,-3]上是\par
A.增函数且最小值为-5 \hfil   B.增函数且最大值为-5  \par
C.减函数且最小值为-5  \hfil  D.减函数且最大值为-5
\end{example}
\begin{example}
下列四个命题:\par
(1)$f(x)=1$是偶函数;\par
(2)$g(x)=x^3,x\in (-1,1]$是奇函数;\par
(3)若$f(x)$是奇函数,$g(x)$是偶函数,则$H(x)=f(x)g(x)$一定是奇函数;\par
(4)函数$y=f(|x|)$的图象关于y轴对称。其中正确的命题个数是(\qquad)\par
A.1    \hfil       B.2   \hfil
        C.3    \hfil           D.4
\end{example}
\begin{example}
已知$f(x)=x^4+ax^3+bx-8$,且$f(-2)=10$,则$f(2)$=\underline{\qquad}
\end{example}
\begin{example}
判断下列函数的奇偶性:\par
(1).$ f(x)=\begin{cases}
 x+1(x>0)\\
1(x=0)\\
-x+1(x<0) \end{cases} $ \underline{\qquad} \par
(3).$f(x)$不恒为0,且对$a,b\in R $恒有$f(a+b)=f(a)+f(b)$\underline{\qquad}

\end{example}
\begin{example}
	已知$f(x)$是定义在$R$上的奇函数当x>0时,$f(x)=\sqrt{x}+1$,则$f(2)$=\underline{\qquad}
\end{example}
\begin{example}
是否存在常数$m,n$,使函数$f(x)=(m^2-1)x^2+(m-1)x+n+2$为奇函数?
\end{example}
\vspace{2.5cm}
\begin{example}
已知函数$f(x)$满足$f(x+y)+f(x-y)=2f(x)f(y)(x,y\in R)$试证明f(x)是偶函数。
\end{example}
\vspace{2.5cm}
\begin{example}
已知函数$y=f(x)$为奇函数,在$(0,+\infty)$内是减函数,且$f(x)<0$,试问:$F(x)=\dfrac{1}{f(x)}$在$(-\infty,0)$内增减性如何?并证明.
\end{example}