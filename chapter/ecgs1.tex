\chapter{二次根式}
\section{二次根式的概念及有意义的条件}
\subsection{概念}
一般地,我们把形如$\sqrt{a}(a\geq 0)$的式子叫做二次根式. $"\sqrt{~~}"$称为二次根号.
$$
\text{两个必备特征} \begin{cases}  
\text{外貌特征:含有$\sqrt{~~}$}\\
\text{内在特征:被开方数$a\geq 0$}
\end{cases}
$$
\subsection{有意义的条件}
\begin{enumerate}
    \item 单个二次根式如$\sqrt{A}$有意义的条件:$A\geq 0$;
    \item 多个二次根式相加如 $\sqrt{A}+\sqrt{B}+\ldots+\sqrt{N}$                         有意义的
    $A \geqslant 0,B \geqslant 0 ,\ldots N \geqslant 0$
    \item 二次根式作为分式的分母如$\frac{B}{\sqrt{A}}$有意义的条件: 
    $A>0$;
    \item 二次根式与分式的和如$\sqrt{A}+\frac{1}{B}$有意义的条件:
    $A \geq 0\text{且} B\neq 0$.


\end{enumerate}
\subsection{例题分析}
\begin{example}下列各式中,哪些是二次根式?哪些不是?\par
(1).$\sqrt{32}$ \hfil (2).6 \hfil (3).$\sqrt{-12}$ \hfil (4).$\sqrt{-m}(\leqslant 0)$\par
(5).$\sqrt{xy}(\text{x,y异号})$  \hfil (6).$\sqrt{a^2+1}$\hfil (7).$\sqrt[3]{5}$
\end{example}
\begin{example}
    当x是怎样的实数时,下列在实数范围内有意义?\par
    (1).$\sqrt{x-2}$\hfil (2).$\dfrac{1}{\sqrt{x-1}}$\hfil (3).$\dfrac{\sqrt{x+3}}{x-1}$\par
    \vspace{2cm}
    (4).$\sqrt{-x^{2}+2 x-1}$ \hfil  (5).$\sqrt{-x^{2}-2 x-3}$
\end{example}
\vspace{2cm}
\subsection{习题}
\begin{enumerate}
    \item  下列各式:$\sqrt{3}, \sqrt{-5},\sqrt{a^{2}} ,\sqrt{x-1}(x \geqslant 1) , \sqrt[3]{27} ,\sqrt{x^{2}+2 x+1}$
    一定是二次根式的个数有 (\qquad)
    \begin{tasks}(4)
        \task 3个
        \task 3个
        \task 5个
        \task 6个
    \end{tasks} \
    \item 若式子$\sqrt{\dfrac{x-1}{2}}$在实数范围内有意义,则x的取值 
    范围是    
    \item   若式子$\dfrac{1}{x-2}+\sqrt{x}$在实数范围内有意义,则x的  
    取值范围是 
    \end{enumerate}
    \section{二次根式的性质}
    二次根式的实质是表示一个非负数(或式)的算术平方根.对于任意一个二次根式$\sqrt{a}$ ,我们知道:
    \begin{enumerate}
       \item a为被开方数,为保证其有意义,可知$a \geq 0$;
       \item $\sqrt{a}$表示一个数或式的算术平方根,可知$\sqrt{a} \geq 0$.
       \item $(\sqrt{a})^{2}=a(a \geqslant 0)$
       \item $\sqrt{a^{2}}=|a|=\left\{\begin{array}{ll}{a} & {(a \geqslant 0)} \\ {-a} & {(a<0)}\end{array}\right.$
   \end{enumerate}
   \subsection{例题分析}
      \begin{example}   
            计算:\par
            (1).$(\sqrt{\dfrac{4}{7}})^2$   \hfil (2).$(4\sqrt{3})^2$\par 
            \vspace{2cm}
            (3). $\sqrt{(-5)^2}$ \hfil  (4).$-\sqrt{(-\dfrac{1}{7})^2}$ 
            \vspace{2cm}  
      \end{example} 
      \begin{example}
          实数$a,b$在数轴上对应的点位置如图所示,化简$|a|+\sqrt{(a-b)^2}$结果为\par
       
             
         \begin{figure}[h]
            \centering
             
       
          \begin{tikzpicture}
          \draw [-Stealth,thick,red](-3cm,0)--(-1cm,0)--(2cm,0);
          \node (0,0) [below]{0};
          \node [below] at (-2cm,0)  {$a$};
          \node [below] at (0.5cm,0) {$b$};
          \fill (-2cm,0) circle (1pt);
          \fill (0.5cm,0) circle (1pt);
          \draw [red,thick](0,0)--(0,3pt);
        
          \end{tikzpicture}
          \caption{\label{fig: } }
        \end{figure}
      \end{example}
      
            \begin{example}
        若$|a-2|+\sqrt{b-3}+(c-4)^{2}=0$,求$a-b+c$的值.
      \end{example}
      \vspace{2cm}
      
      \begin{example}
        已知$y=\sqrt{x-3}+\sqrt{3-x}+8$,求$3 x+2 y$的算术平方根.
      \end{example}
      \vspace{2cm}
      \begin{example}
        已知a,b为等腰三角形的两条边长,且a,b满足$b=\sqrt{3-a}+\sqrt{2 a-6}+4$,求此三角形的周长
      \end{example}
      \vspace{1.7cm}
      \subsection{练习}
      \begin{enumerate}
       \item  已知$|3 x-y-1|$和$\sqrt{2 x+y-4}$互为相反数,求$x+4 y$的平方根.
       \item \vspace{1.7cm} 下列式子中,不属于二次根式的是
       \begin{tasks}(4)
           \task $\sqrt{5}$
           \task $\sqrt{a^{2}}$
           \task $\sqrt{-7}$
           \task $\sqrt{\frac{1}{2}}$
        \end{tasks}
           \item  式子$\dfrac{-2}{\sqrt{3 x-6}}$有意义的条件是  
    \begin{tasks}(4B)
    \task $x>2$
    \task $x \geq 2$
    \task $x<2$
    \task $x \leqslant 2$
    \end{tasks}      
       
       \item 当a是怎样的实数时,下列各式在实数范围内有意义?   \par
       (1)$\sqrt{a-1}$ \hfil (2)$\sqrt{2 a+3}$ \hfil (3)$\sqrt{-a}$ \hfil (4)$\sqrt{\dfrac{2}{5-a}}$
       \item  若二次根式$\dfrac{\sqrt{m-2}}{|m^{2}-m-2|}$有意义,求m的取值范围
        \item  \vspace{2cm}  无论x取任何实数,代数式$\sqrt{x^{2}+6 x+m}$都有意义,求m的取值范围.
        \item \vspace{2cm}  若x,y是实数,且$y<\sqrt{x-1}+\sqrt{1-x}+\frac{1}{2}$,求 $\dfrac{|1-y|}{y-1}$       的值
        \item \vspace{2cm}  当x为何值时,$\sqrt{x(x-1)}$有意义?
        \item \vspace{2cm}  当x为何值时,$\sqrt{\dfrac{x-2}{2 x+1}}$有意义?
        \item one by ong
    \end{enumerate}
      

                     



