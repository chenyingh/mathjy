\chapter{点、直线、平面之间的位置关系}
\section{空间点、直线、平面之间的位置关系}
\subsection{平面}
\begin{theorem}{公理}{31}
	\begin{enumerate}[noitemsep]
	\item 如果一条直线上的两点在一个平面内,那么这条直线在此平面内.\\
	$A\in l,B\in l,且A\in \alpha ,B\in \alpha  \Rightarrow l\subset \alpha $
	\item 过不在一条直线上的三个点,有且只有一个平面
	\item 如果两个不重合的平面有一个公共点,那么他们有且只有一条过该点的公共直线.\\
	$P\in \alpha ,且 P\in \beta \Rightarrow \alpha \cap \beta =l ,且 P\in l  $
\end{enumerate}
\end{theorem}
\subsection{空间中直线与直线之间的位置关系}

\begin{theorem}{公理}{31}
\begin{enumerate}[noitemsep]
	\item  平行于同一条直线的两条直线相互平行(平行线的传递性).
	\item 空间中如果两个角的两边分别对应平行,那么这两个角相等或者互补.
\end{enumerate}
\end{theorem}
\subsection{空间中直线与平面之间的位置关系}
\begin{enumerate}
\item	直线在平面内$\textemdash$ 有无数个公共点.
\item     直线与平面相交$\textemdash $ 有且只有一个公共点.
\item    直线与平面平行$\textemdash $  无公共点.
\end{enumerate}
\subsection{平面与平面之间的位置关系}
\begin{enumerate}
	\item 两个平面平行$\textemdash $没有公共点.
	\item 两个平面相交$\textemdash $有一条直线.
\end{enumerate}
\section{直线、平面平行的判定及其性质}
\subsection{直线与平面平行的判定}
\begin{theorem}{}{}
	平面外一条直线与此平面内的一条直线平行,这该条直线与此平面平行(线线平行$\Rightarrow$)线面平行.
\end{theorem}
\subsection{平面与平面平行的判定}
\begin{theorem}{}{}
	一个平面的内的两条相交直线与另一个平面平行,则这两个平面相互平行(线面平行$\rightarrow$面面平行)。
\end{theorem}
\subsection{直线与平面平行的性质}
\begin{theorem}{}{}
	一条直线与一个平面平行,则过这条直线的任意平面与该平面的交线与该直线平行.
\end{theorem}
\subsection{平面与平面平行的性质}
\begin{theorem}{}{}
	如果两个平行平面同时和第三个平面相交,那么他们的交线平行.
\end{theorem}
\section{直线、平面垂直的判定及性质}
\subsection{直线与平面垂直的判定}
\begin{theorem}{}{}
	一条直线与一个平面内的两条相交直线垂直,则该直线与此平面垂直.
\end{theorem}
\begin{theorem}{}{}
	一个平面过另一个平面的垂线,则这两个平面垂直.
\end{theorem}
\subsection{直线与平面垂直的性质}
\begin{theorem}{}{}
	垂直于同一个平面的两条直线平行.
\end{theorem}
\subsection{平面与平面垂直的性质}
\begin{theorem}{}{}
	两个平面垂直,则一个平面内垂直于交线的直线与另一个平面垂直.
\end{theorem}