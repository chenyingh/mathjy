\chapter{抽屉原理}
\section{知识点}
有3 本书,放在甲乙两个抽屉里,放的方法有以下几种:\\
甲:3,0,2,1\\
乙:0,3,1,2\par
从以上四种情况可以发现:至少有1 个抽屉放了两本或两本以上的书。\par
这就是抽屉原理的一个例子。同样,如果有3 个抽屉,放4 本或多于4 本书,
至少有1 个抽屉放2 本或2 本以上的书。那么到底什么是抽屉原理呢?
\par
\emph{抽屉原理(1):}把n+1 个物体(或多于n+1 个物体),放入n 个抽屉里去,
那么必有一个抽屉里至少放入2 个物体。\par
\emph{抽屉原理(2):}把m$\times$n+1 个物体(或多于m$\times$n+1 个物体),放入n 个抽屉里
去,那么必有一个抽屉里至少放入了m+1 个物体。
\section{例题分析}
\begin{example}
    1999 年1 月出生的任意32 个孩子中,至少有两个人是同一天出生的。
\end{example}
\vspace{3cm}
\begin{example}
    据说人的头发不超过20 万根,如果陕西省有3645 万人,根据这些数据,你
知道陕西省至少有多少人头发根数一样多吗?
\end{example}
\vspace{3cm}
\begin{example}
    一个口袋中有100 个球,其中红球有28 个,绿球有20 个,黄球有12 个,蓝
球20 个,白球10 个,黑球10 个,从袋中任意摸出球来,如果要使摸出的球中,至
少有12 个球颜色相同,那么从袋中至少要摸出多少个球来?
\end{example}
\newpage
\begin{example}
    六年级有41 名同学,他们做了210 只纸鹤,要把这些纸鹤分给全班的学生,是
    否有人一定能分得到6 只纸鹤?
\end{example}
\vspace{3cm}
\begin{example}
    一个袋子里有一些球,这些球仅只有颜色不同。其中红球10 个,白球9 个,黄
    球8 个,蓝球2 个。某人闭着眼睛从中取出若干个,试问他至少要取多少个球,才
    能保证至少有4 个球颜色相同?
\end{example}
\vspace{3cm}
\begin{example}
    五年级有47 名学生参加一次数学竞赛,成绩都是整数,满分是100 分。已知3
名学生的成绩在60 分以下,其余学生的成绩均在75~95 分之间。问:至少有几名
学生的成绩相同?
\end{example}
\vspace{3cm}
\begin{example}
    在口袋里放着红、蓝、黄三种颜色的小球若干个,如果有45 个人从袋子里摸取
    小球,每人只准取2 个小球,那么这45 个人中,至少有多少人摸取的球的颜色情形
    是一样的(不考虑摸出球的顺序)?
\end{example}
\vspace{3cm}
\begin{example}
    夏令营组织200 名营员活动,其中有爬山、参观博物馆和到海滩游玩三个项目。
    规定每人必须参加一项或两项活动。那么至少有几名营员参加的活动项目完全相
    同?
\end{example}
\vspace{3cm}
\begin{example}
    把104 块糖分给14 个小朋友,如果每个小朋友至少分得一块糖的话,那么不
    管你怎样分,一定会有两个小朋友分到的糖块数一样多,为什么?
\end{example}
\vspace{3cm}
\begin{example}
    100 名少先队员选大队长,候选人是甲、乙、丙三人,选举时每人只能投票
    选举一人,得票最多的人当选,开票中途累计,前61 张选票中,甲得35 票,乙得
    10 票,丙得16 票。那么,在尚未统计的选票中,甲至少再得多少票就一定能当选?
\end{example}
\vspace{3cm}
\begin{example}
    100 名少先队员选大队长,候选人是甲、乙、丙三人,选举时每人只能投票
    选举一人,得票最多的人当选,开票中途累计,前61 张选票中,甲得35 票,乙得
    10 票,丙得16 票。那么,在尚未统计的选票中,甲至少再得多少票就一定能当选?
\end{example}