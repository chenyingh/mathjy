\chapter{按比例分配}
\section{知识点}
把一个总量按照一定的比分成若干个分量的应用题,叫做按比例分配。按比例
分配的方法是,将按已知比分配变为按份数分配,把比的各项相加得到总份数,各
项与总份数之比就是各个分量在总量中所占的分率,由此可求得各个分量。\par
解答按比例分配应用题的步骤是\par
第一:求出按比例分配的总数量;\par
第二:找出分配的比,并求各个部分占总数量的几分之几;\par 
第三:用总数量乘以部分量占总数的几分之几得到各部分量。
\section{例题分析}
\begin{example}
    用60 厘米长的铁丝围成一个三角形,三角形三条边的比是3∶4∶5。三条边
    的长各是多少厘米? 
\end{example}
\vspace{3cm}
\begin{example}
    学校把植树560 棵的任务按人数分配给五年级三个班,已知一班有47 人,二
班有48 人,三班有45 人,三个班各植树多少棵?
\end{example}
\vspace{3cm}
\begin{example}
    某工厂第一、二、三车间人数之比为8∶12∶ 21,第一车间比第二车间少80
人,三个车间共多少人?
\end{example}
\vspace{3cm}
\section{练习}
\vspace{-0.4cm}
\begin{exercise}
    建筑工人用水泥、沙子、石子按2: 3:5 配制成96 吨的混凝土,需要水泥、沙
    子、石子各多少吨?  
\end{exercise}
\vspace{3.2cm}
\begin{exercise}
    把280 棵树苗栽在两块长方形地上,一块长15 米,宽8 米;另一块长12 米,宽
4 米,如按面积大小分配栽种,这两块地分别要栽多少棵?
\end{exercise}
\vspace{3.2cm}
\begin{exercise}
    甲、乙两个建筑队原有水泥的重量比是4:3。当甲队给乙队54 吨水泥后,甲、
    乙两队的水泥的重量比是3:4。原来甲队有水泥多少吨?
\end{exercise}
\vspace{3.2cm}
\begin{exercise}
    某农场把61600 亩耕地进行规划,其中粮田与棉田的比是7: 2,棉田与其他作
物田的比是6:1,每种耕地各有多少亩?
\end{exercise}
\vspace{3.2cm}
\begin{exercise}
    王晓峰的书架有上、中、下三层。上层存书本数与存书总数的比是5:21。如果
从下层拿18 本书放到上层,则每层书架的存书本数相等。这个书架共有存书多少
本?
\end{exercise}
\vspace{3.2cm}
\begin{exercise}
    甲、乙两个工地上原来水泥袋数的比是2:1,甲地用去125 袋后,甲、乙两工
地水泥袋数的比为3:4,甲、乙两工地原有水泥多少袋?
\end{exercise}
\vspace{3.2cm}
\begin{exercise}
    纸箱里有红绿黄三色球,红色球的个数是绿色球的$\frac{3}{4}$ ,绿色球的个数与黄色
球个数的比是4: 5,已知绿色球与黄色球共81 个,问三色球各有多少个?
\end{exercise}
\vspace{3.2cm}
\begin{exercise}
    从前有个牧民,临死前留下遗言,要把17 只羊分给三个儿子,大儿子分总
数的$\frac{1}{2}$ ,二儿子分总数的$\frac{1}{3}$ ,三儿子分总数的
$\frac{1}{9}$,并规定不许把羊宰割分,求三个
儿子各分多少只羊?
\end{exercise}
\vspace{3.2cm}
