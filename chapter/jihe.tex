\chapter{集合的表示与运算}
\section{集合的有关概念}
\begin{enumerate}
	\item 集合与元素.
	\item 元素的特征:确定性、互异性、无序性
	\item 集合的分类:\begin{enumerate}
		\item 有限集、无限集
		\item 空集、单元素集
		\item 可数集、不可数集
	\end{enumerate}
	\item 集合的表示:\begin{enumerate}
		\item 描述法:$ \{x|1<2<5\} $
		\item 例举法:\{1,2,3,4,5,6,7\}
		\item 韦恩图法
		\item 区间法  
	\end{enumerate}
	\item 子集与真子集
	\item 常用数集 ($\mathbb{Z}$ 、$\mathbb{R}$、$\mathbb{N}$、$\mathbb{N^*}$、$\mathbb{Q}$)
\end{enumerate}
\section{集合的运算}
\begin{enumerate}
	\item 交集
	\item 并集
	\item 补集
\end{enumerate}
\section{集合的运算律}
\section{例题}
\begin{example}
已知$A=\{x|2\leq x \leq 6\},B=\{x|2a\leq x\leq a+3\}$,若$B\subseteq A$,求实数a的取值范围.\\
\vspace{2cm}
\end{example}
\begin{example}
	在$\mathbb{R}$上定义运算$\oplus:x\oplus y=\frac{x}{2-y}$,关于x的不等式$x\oplus (x+1-a)>0$的解集是[-2,2]的子集,求实数a的取值范围.
\end{example}
\begin{example}
	$M=\{x\in \mathbb{R}|x^2+1=0.\}$ a=0,则下列关系式成立的是(\qquad)\par
	$A.a\in M$ \hfil $B.a\nsubseteq M$ \hfil  $C.\{a\}\supsetneqq M $ \hfil $D.\{a\}\subseteq M $
\end{example}
\begin{example}
$A=\{x|x=a^2+1,a\in N_+\}$;$B=\{x|x=b^2-4b+5,b\in N_+\}$,则下列关系式成立的是(\quad)\par
$A.A=B$  \hfil $B.A\subsetneqq B$ \hfil $C.A\supseteq B$ \hfil  $D.A\subseteq B$
\end{example}
\begin{example}
	$M=\{x,xy,\sqrt{x-y}\},N=\{0,|x|,y\}$若$M=N$,则$(x+y)+(x^2+y^2)+\cdots +(x^{100}+y^{100})=$(\qquad).\par
	$A.-200$  \hfil  $B.200$   \hfil  $C.-100$ \hfil  $D.0$
	
\end{example}
\begin{example}
$M=\{x|x=t^2,t\in Z\}$,$N=\{x\in R||x|<5\}$,则$M\cap N$的所有不同子集共有.(\quad)\par
$A.4$个\hfil  $B.7$个  \hfil    $C.8$个  \hfil    $D.10$个
\end{example}
\begin{example}
	$A=\{x|x=2n+1,n\in Z\}$,$B=\{x|x=4n\pm 1,n\in Z\}$,则下列关系式成立的是()\par
	$A.A=B$ \hfil $B.A\subseteq B$  \hfil  $C.A\supseteq B$ \hfil  $D.A\bigcap B=\emptyset $
\end{example}
\begin{example}
设$A\subset B$,则必为空集的是(\quad)\par
	$A.A\bigcap (\complement _U B)$ \hfil $B.B\bigcap (\complement _U A)$ \hfil  $C.(\complement _U A)\bigcap (\complement _U B)$ \hfil $D.A\bigcap B$
\end{example}
\begin{example}
	$A=\{x\in |mx+n \neq 0,m\neq 0\}$,$B=\{x\in R|px+q\neq 0\}$,则$\{x|(mx+n)(px+q)=0\}=$()\par
	$A.A\bigcap B$  \hfil   $B.A\bigcup B$ \hfil   $C.(\complement _R A)\bigcup (\complement _R B)$  \hfil   $D.(\complement _R A)\bigcap (\complement _R B)$
\end{example}
\begin{example}
	满足$\{1,3\}\bigcup B =\{1,3,5\}$的不同B的个数是(\quad).\par
	$A.1$  \hfil  $B.2$ \hfil   $C.3$ \hfil     $D.4$
\end{example}
\begin{example}	
	若$A\subseteq B \subseteq C$,集合A含3个元素,集合C含6个元素,则不同的集合B共有(\quad)\par
	$A.3$  \hfil  $B.4$ \hfil   $C.6$ \hfil     $D.8$
\end{example}
\begin{example}
	设全集$U=\{(x,y)|x,y\in R\},M=\{(x,y)|\frac{y-3}{x-2}=1\},N={(x,y|y\neq x+1)}$,则$\complement _U (M\bigcup N)=$(\quad).\par
	$A.\emptyset$  \hfil  $B.\{(2,3)\}$ \hfil   $C.(2,3)$ \hfil     $D.\complement _U N$
\end{example}
\begin{example}
	设$A=\{x\in R| x^2+px +q=0\},B=\{x\in R |x^2-px-2q=0\}$,若$A\bigcap B=\{-1\}$,求$A\bigcup B$.
\vspace{3cm}
\end{example}
\begin{example}
	设全集$U=\{1,2,4,a^2-a+1\},A=\{1,3a-2\},B={3,4}$,求$A\bigcap B$.
\vspace{3cm}
\end{example}
\begin{example}
	设$A=\{2,a^2-2a,6\},B={2,2a^2,3a-6}$,若$A\bigcap B=\{2,3\}$,求$A\bigcup B$.
\vspace{3cm}
\end{example}
\begin{example}
	设全集$U=\{x\in N_+ | x \leq 8 \}$,若$A\bigcap (\complement _U B)=\{1,8\}$,$(\complement _U A)\bigcap (\complement _U B)=\{4,7\}$,$(\complement _U A)\bigcap B=\{2,6\},$求集合A,B.
\vspace{3cm}
\end{example}
\begin{example}
	设$A=\{x|0<x<\leq 2\},B=\{x|ax^2-x+1=0\}$,若$\emptyset\subsetneqq B\subsetneqq R$,求证$A\bigcap B\neq \emptyset $
\vspace{3cm}
\end{example}
\begin{example}
	已知$A=\{1,3,x\},B=\{1,x^2\}$,若$B\bigcup (\complement _U B)=A$,求$\complement _U B$
	\vspace{3cm}
\end{example}
\begin{example}
	已知集合$A=\{x|x^2+px+q=0\},B=\{x|qx^2+px+1=0\},$若$A\bigcap B\neq \emptyset$且$A\bigcap (\complement _R B)=\{-2\}$,其中$p,q\neq 0$,求$p,q$的值.
\end{example}